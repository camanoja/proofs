\documentclass[../proofs.tex]{subfiles}

\newcommand{\ftjustified}{\footnote{We will not attempt to define the term
 ``justified", except for loosely requiring that the deduction makes sense,
taking the prior statements into consideration.}}

\begin{document}
\chapter{Bad Proofs}
Here, we explore examples of what \textbf{not} to do, and what consistutes bad
form. Wherever possible, we correct the mistakes made in our first try.

\section{Types of Bad Form}
As we have previously stated, proofs are the power of reasoning and so our
proofs are only as strong as our reason. No matter what proof style we use or
how eloquently we present our proof, if one deduction is faulty then the whole
proof is rejected. The mathematical court holds zero reservations.

The issue boils down to logical validity and circularity. An argument
that is logically valid means that every deduction is justified by the
definitions or deductions \emph{preceding} it.\ftjustified On the other hand,
a circular argument is one that assumes that its conclusion is true. Though a
circular argument is logically valid, its discoveries hold value until the
conclusion is actually proven to be true.

We will explore these cases of bad proofs:
\begin{enumerate}
  \item Inexhaustive Cases,
  \item Deductive Inconsistency,
  \item Circular Logic, and
  \item Ill-definedness.
\end{enumerate}

\section{Inexhaustive Cases}
Consider the following (erroneous) claim:
\begin{claim}
  Let $n \leq 10$, $n \in \naturals$. The number $n! = n \cdot (n-1) \cdots 2
  \cdot 1$ is \textbf{not divisible} by 81.
\end{claim}
\begin{proof}
  Since all the factors of $n!$ for $n < 10$ are factors in $10!$, it follows
  that $10!$ is not divisible by 81, then the claim is true.

  Notice that
  $$10! = 1 \cdot 2 \cdot 3 \cdot 4 \cdot 5
  \cdot 6 \cdot 7 \cdot 8 \cdot 9 \cdot 10.$$

  None of these integer factors are a multiple of 81, as wanted.
\end{proof}

This ``proof" is invalid because it did not check all the \emph{possible}
factors that come from combining the integer factors of $10!$. We show
that, in fact, $9!$ is divisible by 81:
\begin{proof}
  \begin{align}
    9! &= 1 \cdot 2 \cdot 3 \cdot 4 \cdot 5 \cdot 6 \cdot 7 \cdot 8 \cdot 9 \\
       &= 3 \cdot 3 \cdot 9 \cdot (2 \cdot 4 \cdot 5 \cdot 2 \cdot 7 \cdot 8),
        & \text{factoring}\\
       &= 81 \cdot (2 \cdot 4 \cdot 5 \cdot 2 \cdot 7 \cdot 8)
  \end{align}
  So, indeed, $9! = 81k$, for some $k \in \naturals$, and thus the above
  claim is false.
\end{proof}

\section{Inconsistency}
Consider the following claim:

\begin{claim}
  0 = 1
\end{claim}
\begin{proof}
  Let $a = 1, b = a$. Then:
  \begin{align}
    &a = b & \text{commutativity of $=$} \\
    \implies &a^2 = b^2 \\
    \implies &a^2 = ab  \\
    \implies &a^2 - b^2 = ab - b^2 \\
    \implies &(a + b)(a - b) = b(a - b) \\
    \implies &a + b = b & \text{dividing by $(a-b)$} \label{eq: baddiv} \\
    \implies &a = 0 \\
    \implies &1 = 0 & \text{$a = 1$, by definition}
  \end{align}
\end{proof}

Of course, the lines from equation (\ref{eq: baddiv}) downward do not hold, as
the operation performed is a division by zero. A division by zero by any number
never yields a finite number, so the deduction is invalid in $\reals$.

\section{Circular Logic}
We incorrectly prove the \textbf{correct} claim:
\begin{claim}
  Let $x, y \in \reals$ such that $0 \leq y \leq x$. Then $y^2 \leq x^2$.
\end{claim}
\begin{proof}
  \begin{align}
    &y \leq x, \text{ by assumption} \label{eq: sqassump} \\
    \implies &y^2 \leq x^2 \text{, squaring both sides.} \label{eq: sqconseq}
  \end{align}
\end{proof}

Perhaps because of its conciseness, this proof may seem to hold at first sight.
However, notice that the claim is 

\section{Ill-definedness}


\end{document}
