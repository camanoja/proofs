\documentclass[../proofs.tex]{subfiles}

\begin{document}
\chapter{Direct Proofs}
 Just as the name implies, a \textbf{direct proof} is a method of proving that
 uses known facts to directly prove a given statement. We might synonymise the
 term ``direct proof" with ``straightforward reasoning".
  
 \begin{expl}{ Cookies }
  Alice, Bill, and Claire love chocolate-chip cookies. With their
  combined efforts, they are able to finish a cookie jar from full
  capacity in very short time. One morning, the three discover that
  one cookie remains in the jar and they decide to work out the
  logistics of sharing it later in the day. They come back that
  afternoon to find that the cookie jar has been releved of its treat.
  Alice asks with genuine curiousity: ``who ate the cookie?'', 
  to which Claire replies: ``I don't know!''. Our job is to figure out
  who ate the cookie.

  Assume that in this scenario, (1) a person speaks with complete sincerity
  and (2) only Alice, Bill or Claire could have eaten the cookie. 
  
  We deduce that Alice couldn't have been the one to eat the cookie, since
  she asked the question and we assumed sincerity. Thus either Bill or Claire
  had eaten the cookie. Just as quickly we find that Claire is innocent
  for the following reason: \emph{if} she did eat the cookie, then she would
  have known who ate the cookie: namely herself. 
  
  Thus the only person left is Bill. Since either Alice, Claire or Bill
  could have eaten the cookie, and we know that neither Alice nor Claire
  did it, we conclude that Bill must have eaten the cookie. 
 \end{expl}


 \begin{expl}{ Simplifying Taxes }
   At the grocery, one finds that the total base price for
   thier items is multiplied by some scaling constant $t$.
   As of writing this document, in Toronto, Canada, $t = 1.13$.
   Suppose that our grocery list is:
  \begin{center}
    \begin{tabular}{|l|c|}
    \hline 
    Item                   & Price \\ 
    \hline 
    Bag of Apples          & 1.75 \\ 
    French Baguette        & 1.25 \\ 
    Bag of Coffee Grounds  & 7.00 \\ 
    \hline 
    \end{tabular} 
  \end{center}

  So, our base price is $b = 1.75 + 1.25 + 7.00 = 10.00$ dollars.
  Thus, our final price is $t \times b = 1.13 (10.00)$ dollars.

  % taxes are associative
  What we really want to know is how much more we are going to pay after
  taxes, so we focus on the $13\%$ corresponding to the decimal $0.13$.
  Mentally computing $13\%$ of $10.00$ is not the easiest task. An 
  easier task is computing $10\%$ of $13.0$ since it involves just moving
  the decimal place 1 place left.
  The numbers that we calculate are the same since by associatvity: 
  $$ t \times b = 0.13 \times 10 = \frac{13}{100} \times 10 = 0.1 \times 13.$$
  

  % distributive
  Next, suppose that we forget to buy items on our shipping list,
  so we buy each item seprately on multiple trips. So the tax is
  applied to each item:

  $$(1.13 \times 1.75) + (1.13 \times 1.25) + (1.13 \times 7.00)$$

  We are concerned with whether buying these items seperately actually
  costs us more than buying them all at once. But we know that by
  factoring out the common coefficient $1.13$:
  $$ (1.13 \times 1.75) + (1.13 \times 1.25) + (1.13 \times 7.00)
  = 1.13 (1.75 + 1.25 + 7.00) $$

  So buying each item individually wastes us no more money than buying the entire
  list at once (time, however, is another story). 
 \end{expl}



 When we formalize our straightforward reasoning, we must also
 formalize our reasoning's structure. The proof structure, in order, is:

 \begin{enumerate}
   \item The statement to be proved (the claim),
   \item Relevant definitions,
   \item Logical deductions that follow from the definitions, and
   \item The final conclusions.
 \end{enumerate}

 So, applying this structure to the first example in this chapter, we have:
 \begin{enumerate}
   \item Bill is the one who ate the cookie,
   \item (1) Everyone in this scenario speaks with sincerity, and 
   (2) Either one of Alice, Bill or Claire could have eaten the cookie.
   \item The deductions that lead us to show that neither Alice nor Claire
    ate the cookie.
   \item Because of (2), Bill must have eaten the cookie.
 \end{enumerate}
 In general, the structure will be assumed implicit so
 we won't enumerate the deductive structure in our proofs.

% TODO: How to start off with a proof?
% TODO: Intuitions, known properties, pattern matching etc

We start off by proving some naturally understood properties of the real numbers.
\inputdpsrc{proofs_with_algebra}
Before we start the next section, it might be worthwhile to note in passing
that, unless the proof is trivial, proofs require us to observe and describe
a phenomenon in more than one way; to glean an additional perspective. And when
we combine our knowledge, we lead ourselves to new truths. This is especially
true when we prove Pythagoras' Theorem next.

\inputdpsrc{appeal_geometry}

\end{document}
