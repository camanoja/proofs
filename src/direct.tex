\documentclass[../proofs.tex]{subfiles}
\begin{document}
\chapter{Direct Proofs}
  A \emph{direct proof} can be
  thought of as using known facts to prove a statement, without resorting to an
  alternative method. We illustrate with a few examples. \\

\begin{expl}
  Prove that the sum of two odd numbers is an even number.

\begin{proof}
% define odd numbers
Let $a$ and $b$ be odd numbers. We may express $a$ and $b$ as:
\begin{align*}
  a &= 2n + 1, \quad \text{ and}   \\
  b &= 2m + 1; \quad \text{for } n,m \in \mathbb{Z}
\end{align*}

This comes from the definitions of odd numbers. As a verifying example,
$a = -7, b = 3$ if $n = -4$ and $m = 1$. Adding $a$ and $b$, we immediately get
the desired result:
\begin{align*}
  a + b &= (2n + 1) + (2m + 1) 	& \justify{by definition of $a$ and $b$} \\
       &= (2n + 2m) + (1 + 1) 	& \justify{by the commutative property of $\mathbb{Z}$} \\
       &= 2(n+m) + 2			& \justify{factoring} \\
       &= 2(n+m+1)				& \justify{factoring}
\end{align*}

Since the sum of an integer and another integer will always be an integer, we conclude that $a+b$, by definition, is an even number, as wanted.

\end{proof}

\end{expl}

In this case, we used the known definitions of odd numbers and properties of
simple algebra to achieve our desired result.
\end{document}
