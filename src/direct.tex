\chapter{Direct Proofs}
  A proof which has a direct proof structure can be thought of as using known facts to prove a statement, without resorting to an alternative method. We provide an example. \\

\noindent \example{Prove that the sum of two odd numbers is an even number.}
\begin{proof}
  Let $a$ and $b$ be odd numbers. Then, we must have
  \begin{align*}
    &a = 2n + 1, \quad \text{ and}   \\
    &b = 2m + 1; \quad \text{for } n,m \in \mathbb{Z}
  \end{align*}
  This is from the definition of odd numbers (note that number $n$ is even if $n = 2k, k \in \mathbb{Z}$.) Adding $a$ and $b$, we immediately retrieve the desired result:
  \begin{align*}
    a + b &= (2n + 1) + (2m + 1) 	& \justification{by definition of $a$ and $b$} \\
         &= (2n + 2m) + (1 + 1) 	& \justification{by the commutative property of $\mathbb{Z}$} \\
         &= 2(n+m) + 2			& \justification{factoring} \\
         &= 2(n+m+1)				& \justification{factoring}
  \end{align*}
  Since the sum of an integer and another integer will always be an integer, we conclude that $a+b$, by definition, is an even number, as wanted.
\end{proof}
One might regard the notion of a direct proof as so basic that it is hard to understand. Perhaps contrasting a direct proof against an indirect proof may bring some clarity.
