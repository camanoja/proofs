\section{Proofs with Algebra}
The tools of algebra give extreme expressive power in our proof structure.

\begin{expl}{Mutually Inclusive Definitions}
The definition of odd and even functions might lead you to believe that odd and
even functions are mutually exclusive. We show that due to one special
function, this is not the case.

% TODO: explicit domain
\begin{claim} The function $z(x) = 0$ is the only function that is both an even
  function and an odd function.
\end{claim}
\begin{proof}
  Let $f$ be some function that has the property that it is both even and odd.
  Then, for all $x \in \dom(f)$, $f$ must also satisfy the following:
  \begin{align}
    2f(x) \label{direct: expl: ffsum1}
      &= f(x) + f(x) \\
      &= f(x) + f(-x) & \justify{$f$ is even} \\
      &= f(x) + (-f(x)) & \justify{$f$ is odd} \\
      &= f(x) - f(x) \\
      &= 0 \label{direct: expl: ffsum2}
  \end{align}
  By the transitive property of equality, we have that expressions
  (\ref{direct: expl: ffsum1}) and (\ref{direct: expl: ffsum2}) are equal. Thus
  $2f(x) = 0$, meaning that $f(x) = 0 = z(x)$ by dividing by 2 on both sides.


  We conclude that if $f$ is both even and odd, then it is necessarily equal
  to $z$, the additive function identity.
\end{proof}

Note a technicality: $z$ must have the property that if $x \in \dom(z)$, then
$-x \in \dom(z)$. For example, the function $z'(x) = 0$ defined
on $x \in [1, 2]$ is neither even nor odd, while $z''(x) = 0$ defined on
$x \in [-2, -1] \cup [1, 2]$ is.
\end{expl}

The example above illustrates the power of a proof: out of the infinite
possibilities and combinations of functions, we assert that only one function
fits the criteria.

\begin{expl}{The sum of two odd numbers is even}
\begin{theorem}
  Let $a$ and $b$ be odd numbers. Then $a + b$ is an even number.
\end{theorem}
\begin{proof}
% define odd numbers
We may express $a$ and $b$ as:
\begin{align*}
  a &= 2n + 1, \quad \text{ and}   \\
  b &= 2m + 1; \quad \text{for } n,m \in \ints
\end{align*}

This comes from the definitions of odd numbers. As a verifying example,
$a = -7, b = 3$ if $n = -4$ and $m = 1$. Adding $a$ and $b$, we immediately get
the desired result:
\begin{align*}
  a + b &= (2n + 1) + (2m + 1) 	& \justify{by definition of $a$ and $b$} \\
        &= (2n + 2m) + (1 + 1)  & \justify{commutativity of $\ints$} \\
        &= 2(n+m+1)				& \justify{simplifying}
\end{align*}

Since the sum of an integer and another integer will always be an integer,
we conclude that $a+b$, by definition, is an even number, as wanted.

\end{proof}

\end{expl}

In this case, we used the known definitions of odd numbers and properties of
simple algebra to achieve our desired result.

% another power: all the infinite combinations of a specific class of objects
%  share a certain property

Aside from finding nice properties of various objects, proofs allow us
to develop algebraic tools:
\begin{theorem}
  Let $a, b \in \reals$ such that $a = b$. Then, for all $c \in \reals$,
  $a + c = b + c$.
\end{theorem}
\begin{proof} We deduce:
  \begin{align}
    a + c &= a + 0 + c \\
          &= a + (b - a) + c, & \justify{since $b = a$} \\
          &= (a - a) + b + c, & \justify{commutativity of $+$} \\
          &= b + c
  \end{align}
  So, by the transitivity of equality, we have $a + c = b + c$, as wanted.
\end{proof}

\begin{theorem}
  Let $a, b \in \reals$ such that $a = b$. Then for all $c \in \reals$,
  $ca = cb$.
\end{theorem}

\begin{proof}
  \begin{align}
    ca &= ca \cdot (1)  \\
       &= ca \cdot (b / a) & \justify{because $b = a$ means $b = 1 \cdot a$} \\
       &= cb \cdot (a / a) & \justify{commutativity of $\cdot$} \\
       &= cb
  \end{align}
  Thus, $ca = cb$, as wanted.
\end{proof}
