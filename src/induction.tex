\documentclass[../proofs.tex]{subfiles}
\begin{document}

\chapter{Mathematical Induction}
The proof technique of Mathematical Induction is powerful and widely used tool

\section{``Induction"}
In daily life, we are exposed to events that strenghthen our conviction of a
statement. For example, imagine that, whenever you pay attention to it, the
morning bus that goes to the subway always arrives at 10:15 AM at the stop
in front of your house. 
Of course, you would have more conviction toward the statement ``The bus always
arrives at 10:15 AM at the stop in front of my house'' after the 500th time the
bus arrives on time as opposed to the 2nd time. This process is induction in
action.

Induction, roughly speaking, refers to the process by which our confidence
in a statement is supported by the number of verifying examples of that
statement. Our confidence increases with the number of examples that support 
our hypothesis.

The following point is critical. Unless we are able to verify our hypothesis
for all (potentially infintite) instances, then we will never achieve complete
certainty of our statements. How could we be sure that tomorrow, next week,
or next year, our bus will be late?

\section{Mathematical Induction}
In a similar vein to the above, consider the statement: for every
$n \in \mathbb{N}, n \geq 4, n^2 < 2^n$. % TODO: prove 
We know that this is true, and one could go about calculating
the values for both $n^2$ and $2^n$, for $n = 1$ and then comparing those
values to verify the predicate, repeating the process for $n=2$, then
$n=3, \ldots$ However, the analogous problem presents itself here: what if
there was some number $M \in \mathbb{N}$ that is beyond the scope of
conceivable human (and computer) discovery\footnote{For example,
Graham's Number}? How do we verify the predicate for such an uncomputably
large number? And even numbers of that magnitude larger?

\section{Proofs by Induction}
  % Define stucture
  The structure of a
  %
\end{document}