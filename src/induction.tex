\documentclass[../proofs.tex]{subfiles}
\begin{document}

\chapter{Mathematical Induction}

\section{``Induction"}
We explore the definition of induction, and use it as motivation for the similarly named proof technique. \\

% CITE
According to Merriam-Webster, Induction is defined as: ``\emph{inference of a generalized conclusion from particular instances}." That is, the validity of a hypothesis is based on the breadth of cases for which said hypothesis does hold.\\

An essential facet of this notion is that one can never be \emph{totally} confident that a certain statement holds. For instance, it may be the case that no Beagle that has ever existed has never naturally grown green fur, but perhaps far in the future there might exist a Beagle whose genetics are such that it does naturally grow green fur. In this way, one may be fairly confident in the statement \emph{``Beagles do not have green fur"}, but never totally confident. \\

 In a similar vein, consider the statement: $\forall n \in \mathbb{N}, n \geq 4, n^2 < 2^n$. Intuitively, this seems true, and one could go about calculating the values for both $n^2$ and $2^n$, for $n = 1$ and then comparing those values to verify the predicate, repeating the process for $n=2$, then $n=3, \ldots$  etc. However, the analogous problem presents itself here: what if there was some number $M \in \mathbb{N}$ that is beyond the scope of conceivable human (and computer) discovery\footnote{For example, Graham's Number}? How do we verify the predicate for such an uncomputably large number? And even numbers of that magnitude larger?

\section{Proofs by Induction}
  % Define stucture
  A
  %
\end{document}