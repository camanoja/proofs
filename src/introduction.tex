\chapter{Introduction}
\section{What are Proofs?}
Given a statement $S$, a proof for $S$ is a convincing argument that shows $S$ is true.
The structure of a proof combines knowledge that we have to \emph{discover} knowledge
that we did not previously know.

Proofs are the cornerstone of theoretical mathematics. Their generative nature
allows us to create powerful tools to generalize, dissect and classify different
phenomena.

This document will discuss a handful of different ways to prove some statement $S$.
We will find that in the same way that there are different approaches to solving a
problem, there are different proof \emph{techniques}, or \emph{styles} which
facilitate the verification or rejection of $S$, while using any other method would
prove to be a daunting task. In this way, there is a certain artistry in the
technique for which someone proves $S$.

% allows us to show some perhaps counter-intuitive relationships
Proofs are particularly important as the product of a successful proof allows us
to definitively state whether some statement is true or false. This allows us to correctly assert
some perhaps counter-intuitive results:
\begin{claim}
  The number $0.999... = 0.\bar{9} = 1$.
  \begin{proof}
    We examine properties of and manipulate the number $0.\bar{9}$ to achieve the desired
    equality. Let $n = 0.\bar{9}$.
    \begin{align}
       n  &= 0.\bar{9} & \justify{Definition}  \\
       10n &= 9.\bar{9}   											\\
       10n - (n) &= 9.\bar{9} - (0.\bar{9}) & \justify{Definition of $n$} \\
       9n &= 9 \\
        n &= 1
    \end{align}
    And thus, we see from the equivalence of the first and last equations that $0.999 = 1$, as claimed.
  \end{proof}
\end{claim}

Note that the term ``convincing" previously stated is extremely important here.
Perhaps the justification for the shifts and cancellations in the 2nd to 4th lines
are not convincing enough - we provide an alternative proof to the claim:
\begin{proof}
  \begin{align}
    3 \cdot \frac{1}{3} &= \frac{1}{3} + \frac{1}{3} + \frac{1}{3} \\
                        &= 0.\bar{3} + 0.\bar{3} + 0.\bar{3} & \justify{Decimal representation of 1/3} \\
                        &= 0.\bar{9}
  \end{align}
  But it is clear that $3 \cdot \frac{1}{3} = \frac{3}{3} = 1$.
\end{proof}

Conceptually speaking, proofs are divided into two groups: \emph{Direct} and \emph{Indirect}.
We review some important concepts of mathematics before diving into these techniques.
