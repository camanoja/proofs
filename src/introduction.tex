\documentclass[../proofs.tex]{subfiles}
\newcommand{\hrulesmall}{\rule{0.4\textwidth}{1pt}}

\begin{document}
\chapter{Introduction}
\section{What are Proofs?}
% Deductive arguments
In logical philosophy (of which mathematics is a subfield), one is often concerned
with whether a statement concerning some object is true or not. Given some fundamental
properties about these objects as a starting point, the task becomes to analyse
and combine these properties to argue various statements about these objects as
a means to \emph{discover} new properties of these objects. A classic example is the
\textit{Knights and Knaves} dilemma.

% Example: Deductive Arguments
\begin{expl}
In the land of Knights and Knaves, knights always tell the truth and
  knaves always tell a lie.

You meet two people, Alice and Bob (henceforth referred to as $A$ and $B$, respectively.)
$A$ says ``Both of us are knaves." Our job is to determine who is a knight and who is a knave.

\begin{list}{}{}
  \item Suppose that $A$ is a knight. If $A$ is a knight, then the statement ``Both of us are knaves" is a lie.
  \item Since knights cannot tell a lie, $A$ must not have been a knight in the first place.
  \item Therefore, we conclude that $A$ is a knave.
  \item Since knaves always tell a lie, then the statement ``Both of us are knaves" has to be false.
  \item The only way for the statement to be false is for $B$ to be a knight. Therefore, $B$ must be a knight.
  \item \hrulesmall
  \item Therefore, $A$ is a knave and $B$ is a knight.
\end{list}
\end{expl}

We see that in the above example, given starting assumptions about knights, knaves and
a statement about $A$, we are able to discern who is a knight and who is a knave.
This style of argument is known as a \emph{deductive argument}.


Given a statement $S$, a \emph{proof} for $S$ is a convincing argument that shows $S$ is true.
The structure of a proof combines knowledge that we have to \emph{discover} facts about that knowledge
that we did not previously know.

Proofs are the heart of theoretical mathematics. The generative nature of a proof
allows us to create powerful tools to generalize, dissect and classify different
mathematical phenomena.

This document will discuss a handful of different ways to prove some statement $S$.
We will find that in the same way that there are different approaches to solving a
problem, there are different proof \emph{techniques}, or \emph{styles} which
facilitate the verification or rejection of $S$, while using any other method would
prove to be a daunting task. In this way, there is a certain artistry in the
technique for which someone proves $S$.

% allows us to show some perhaps counter-intuitive relationships
The product of a (successful) proof allows us to definitively state whether some
statement is true or false. This allows us to assert some perhaps counter-intuitive results:
\begin{claim}
  The number $0.999... = 0.\bar{9} = 1$.
  \begin{proof}
    We examine properties of and manipulate the number $0.\bar{9}$ to achieve the desired
    equality. Let $n = 0.\bar{9}$.
    \begin{align}
       n  &= 0.\bar{9} & \justify{Definition}                             \\
       10n &= 9.\bar{9}   											                          \\
       10n - (n) &= 9.\bar{9} - (0.\bar{9}) & \justify{Definition of $n$} \label{intro: eq: nsubstract} \\
       9n &= 9                                                            \label{intro: eq: 9nwhole}\\
        n &= 1
    \end{align}
    And thus, we see from the equivalence of the first and last equations that $0.999 = 1$, as claimed.
  \end{proof}
\end{claim}
We note that the step from \ref{intro: eq: nsubstract} to \ref{intro: eq: 9nwhole} \emph{require} that the decimal representation is non-terminating.


Note that the term ``convincing" previously stated is extremely important here.
Perhaps the justification for the shifts and cancellations in the 2nd to 4th lines
are not convincing enough - we provide an alternative proof to the claim:
\begin{proof}
  \begin{align}
    3 \cdot \frac{1}{3} &= \frac{1}{3} + \frac{1}{3} + \frac{1}{3} \\
                        &= 0.\bar{3} + 0.\bar{3} + 0.\bar{3} & \justify{Decimal representation of 1/3} \\
                        &= 0.\bar{9}
  \end{align}
  But it is clear that $3 \cdot \frac{1}{3} = \frac{3}{3} = 1$.
\end{proof}

Conceptually speaking, proofs are divided into two groups: \emph{Direct} and \emph{Indirect}.
We review some important concepts of mathematics before diving into these techniques.
\end{document}
