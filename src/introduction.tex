\documentclass[../proofs.tex]{subfiles}
\newcommand{\hrulesmall}{\rule{0.4\textwidth}{1pt}}
\newcommand{\assump}{\textbf{[A]}}
\newcommand{\deduct}{\textbf{[D]}}

\newcommand{\ftfnset}{
Formally, a function
}

\begin{document}
\chapter{Introduction}
\section{Goals of this Booklet}
The main goal of this booklet is to provide a clear introduction to
logical proof structure - a way of formalizing reasoning. I hope that by the end of the
booklet, you have gained a few mathematical tools to use to prove various things.
I aim to do this by providing a large breath of examples throughout that involve familiar math
involving numbers, basic geometry, and a bit of calculus.

I write this booklet as a first-glance introduction to formal proof structure,
\textbf{primarily} for readers who have not had prolonged exposure to this set
of techniques before. To be sure, I would not consider this booklet to be a
textbook in any sense, but rather as a pocket guide for utilizing some of the
most powerful tools in math. In this way, I aim for a balance of rigour: as
much of it that does not distract away from the fundamental idea of the proof
itself, which is the technique being used. For example, when we speak about
functions, we will be thinking about them as ``mathematical machines" that
take a number as its input and returns a unique number as its output - rather
than the more fundamental set formulation.

% structure of the booklet
% each section starts with a few examples, ending with a capstone example
To be sufficiently concise, this paper must assume a few things. In particular,
this paper asks for a familiarity of high school mathematics: that in the realm
of algebra, factoring, functions, variables, et cetera. A section that
defines the preliminary requirements is provided in the next chapter.

% on the language used
Proofs in mathematics consist of symbolic relationships (e.g. expressions and
equations) and natural language to describe the logic being applied. As is a
feature of math, the way that we provide these descriptions require a certain
structure. This fact is useful to keep in mind as we explore different examples
in this booklet.

\section{Why?}
Even a quick search on the web gives handfuls of professional resources for math learning. So a
question that I have anticipated while I have been writing this booklet is ``\emph{Why make
another?}'' My answer to this is that though there are many resources written by professionals
and instructors who \emph{were} once students, this resource is written by someone who \emph{is} a
student. I think that, counter-intuitively, my position affords me a few advantages. I believe
that I can relate my thoughts and ideas more naturally to people who are around my skill range:
the manner of speaking and relaying thoughts among children is much more fluid than when
interacting with other age groups and I think the analogy carries here. 


Since I am not writing this expecting professional gain, I have a larger freedom to illustrate 
mathematical ideas at a more intuitive level and choose to be more rigourous at will; it is easier
see the forest in spite of the trees. For an example, while it is true that functions can be
rigourously defined in terms of ``relations" which can be rigourously defined in terms of ``sets''
(whatever these things mean), it is important that we can still \emph{interpret} what a function
is: a ``machine'' that spits out a unique output for every input. It is important that despite
definitions, we still understand how to manipulate them. For example, we can always add the output
of two functions to create a new function, we can multiply them, we can take their derivatives,
etc.


Finally, since this is \emph{not} a textbook, there are no assignments. There is no pressure to
get a good mark, no pressure to understand every single little detail in the book and no pressure
to cram. This booklet was written for the express love of math and learning, and if it inspires you
then may you continue on the path of more complex and beautifully rigourous math.

\section{Transparency}
%As you are reading this booklet you may be - and you have the right to be - questioning whether 
As is the case in any learning endeavour, one is bound to make mistakes. Being able to recognize
a flaw in a piece of work and correcting it to make the work better allows us to grow. Mathematics
is no different. While perhaps we will make a mistake or two in an exercise or in a proof, I will
admit to some mistakes in writing that I have subsequently tried to correct in the hope that this
product is all the more better for it:

\begin{itemize}
  \item In the first stages of writing this booklet I have written it for an imaginary audience
  who already knows of the content I am writing about. Of course this defeats the purpose of the
  booklet, so I have revised my mindset in writing this current version.

  \item The starting point of the booklet should not be what is the most elementary, but what is
  most \emph{natural} and \emph{familliar}. This way, when we are more aquatinted to proofs we can
  tackle these harder elementary proofs.

  \item I had started with an obscene amount of prerequisites for the book as opposed to naturally
  defining them by example. This prerequisite knowledge hurts the reachability of the booklet
  so I have endeavoured to have the minimum amount of prerequisite knowledge.
\end{itemize}

I believe being transparent here is important: it forms us a common ground where we can talk about
mathematics without fear.


\section{Enter Reasoning}
It is important to answer a few questions about the nature of proofs before we
get to proving.

\paragraph{Who Proves?} In daily life, we are bombarded with questions
about the world that need answering: ``Is it raining today?", ``Will the bus
arrive at 10:15 this morning?", ``Do I meet the qualifications for this job?"
et cetera. Questions like these vary in complexity and impact on our lives,
and in most cases we want to be right in our conclusions, especially when these
judgments affect our decisions on other questions down the line. Thus, in this
way, those who want to understand are those who prove.

\paragraph{When to Prove?} Whenever we are curious about some phenomenon,
it is natural to ask questions about its nature. Finding answers to these
questions increases our understanding; it shines light on that phenomenon.
Our answers inform how we interact with these objects, and determine other
questions we might ask about them. Thus, we prove whenever we are curious.

\paragraph{What are Proofs?} Proofs are the art of reason. A proof
requires the application of logic to knowledge to to discover new facts about
the subject at hand. There is a tangible power to a proof: it \textbf{asserts}
unequivocally that something is true; its veracity is independent of time,
space, language or culture - conditional on that it is ``correct". A
proof distinguishes what is true and what is false.


Proofs lend themselves especially well to mathematics in large part because the
subject is so pure. Mathematical definitions are necessarily unambiguous and the
investigation process follows a strict procedure where each step must be
unchallengable. However, proofs are not drab things. As we will see, these
rules we give ourselves will give life to beautiful scenery.

\paragraph{Why Prove?} To prove is to wield the scepter of reason. We engage
reason so that we can answer questions. We answer questions so that we can
direct our decisions. We prove because we are naturally curious, and to be
curious is to be human.

That might strike as too mystical or ethereal. If so, then notice that many
scientific advancements in physics, chemistry, biology and even the humanities
are solidly based upon logical proofs. The effect of proofs surrounds us.
Devices such as our phones that allow us to interact with each other from far
distances instantaneously or structures such as bridges that stand the test of
time or technology that sends people to the moon all rely on concepts
rigourously proven; guaranteeing that these things will work. Thus, proofs
empower the fields that use them.

%\paragraph{Are proofs hard?} 
%You could argue that there are two arguments to proving something. 
%The first thing is finding out to 

\section{Motivation}
Is it necessarily true that you need to have some insight of the divine to discover mathematical
truths? I would argue that this is not the case at all: you just need a starting point plus a 
little curiousity. To see what I mean, let's add up the odd numbers in order starting from 1:

\begin{align*}
  &1 = 1 \\
  &1 + 3 = 4 \\
  &1 + 3 + 5 = 9 \\
  &1 + 3 + 5 + 7 = 16 \\
  &... 
\end{align*}
Do you see a pattern? How about when we reverse the equalities?
\begin{align*}
  &1 = 1 \\
  &4 = 1 + 3 \\
  &9 = 1 + 3 + 5 \\
  &16 = 1 + 3 + 5 + 7 \\
  &25 = 1 + 3 + 5 + 7 + 9 \\
  &... 
\end{align*}

It seems we can conjecture that \emph{adding consecutive odd numbers in order, we can generate
consecutive squares of integers}. Let's see if we can convince ourselves of this. Since we have
perfect squares on the left-hand-side (LHS) of our equations, it is reasonable to analyze the
structure of a square. 

Well, the most basic square is the 1x1 square. Lets give this square a name, say, $S_1$. 
We know the area is the product of the side lengths: 
$1 \times 1 = 1$. But how do we connect this to the square $S_2$ of side length 2 and area
$2 \times 2 = 4$? Lets try \emph{extending} $S_1$. If we add two 1x1 rectangles to the left and
bottom of $S_1$, we have






\paragraph{} For the rest of this document, we turn to answer the question:
\textbf{How to Prove?}

\end{document}
