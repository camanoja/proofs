\documentclass[../proofs.tex]{subfiles}
\newcommand{\hrulesmall}{\rule{0.4\textwidth}{1pt}}
\newcommand{\assump}{\textbf{[A]}}
\newcommand{\deduct}{\textbf{[D]}}

\newcommand{\ftfnset}{
Formally, a function
}

\begin{document}
\chapter{Introduction}
\section{Goals of this Booklet}
The main goal of this booklet is to provide a clear and concise introduction to
logical proof structure - a way of formalizing reasoning. By the end of the
booklet, my hope is that you have gained a few mathematical tools to use to
prove various things. I aim to do this by providing a large breath of examples
throughout that involve familiar math involving numbers, basic geometry, and a
bit of calculus.

I write this booklet as a first-glance introduction to formal proof structure,
\textbf{primarily} for readers who have not had prolonged exposure to this set
of techniques before. To be sure, I would not consider this booklet to be a
textbook in any sense, but rather as a pocket guide for utilizing some of the
most powerful tools in math. In this way, I aim for a balance of rigour: as
much of it that does not distract away from the fundamental idea of the proof
itself, which is the technique being used. For example, when we speak about
functions, we will be thinking about them as ``mathematical machines" that
take a number as its input and returns a unique number as its output - rather
than the more fundamental set formulation.

% structure of the booklet
% each section starts with a few examples, ending with a capstone example
To be sufficiently concise, this paper must assume a few things. In particular,
this paper asks for a familiarity of high school mathematics: that in the realm
of algebra, factoring, functions, variables, et cetera. A section that
defines the preliminary requirements is provided in the next chapter.

% on the language used
Proofs in mathematics consist of symbolic relationships (e.g. expressions and
equations) and natural language to describe the logic being applied. As is a
feature of math, the way that we provide these descriptions require a certain
structure. This fact is useful to keep in mind as we explore different examples
in this booklet.

\section{Enter Reasoning}
It is important to answer a few questions about the nature of proofs before we
get to proving.

\paragraph{Who Proves?} In daily life, you and I are bombarded with questions
about the world that need answering: ``Is it raining today?", ``Will the bus
arrive at 10:15 this morning?", ``Do I meet the qualifications for this job?"
et cetera. Questions like these vary in complexity and impact on our lives,
and in most cases we want to be right in our conclusions, especially when these
judgments affect our decisions on other questions down the line. Thus, in this
way, those who want to understand are those who prove.

\paragraph{When to Prove?} Whenever we are curious about some phenomenon,
it is natural to ask questions about the nature of that phenomenon. Finding
answers to these questions shines light on our understanding of that phenomenon.
Our answers inform how we interact with these objects, and determine other
questions we might ask about them. Thus, we prove whenever we are curious.

\paragraph{What are Proofs?} Proofs are the artistry of reason. A proof
requires the application of logic to knowledge to to discover new facts about
the subject at hand. There is a tangible power to a proof: it \textbf{asserts}
unequivocally that something is true; its veracity is independent of time,
space, language or culture - conditional on that it is ``correct". A
proof distinguishes what is true and what is false.


Proofs lend themselves especially well to mathematics in large part because the
subject is so pure. Mathematical definitions are necessarily unambiguous and the
investigation process follows a strict procedure where each step must be
unchallengable. However, proofs are not drab things. As we will see, these
rules we give ourselves will give life to beautiful scenery.

\paragraph{Why Prove?} To prove is to wield the scepter of reason. We engage
reason so that we can answer questions. We answer questions so that we can
direct our decisions. We prove because we are naturally curious, and to be
curious is to be human.


\paragraph{} For the rest of this document, we turn to answer the question:
\textbf{How to Prove?}

\end{document}
