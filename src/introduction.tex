\documentclass[../proofs.tex]{subfiles}
\newcommand{\hrulesmall}{\rule{0.4\textwidth}{1pt}}
\newcommand{\assump}{\textbf{[A]}}
\newcommand{\deduct}{\textbf{[D]}}

\begin{document}
\chapter{Introduction}
\section{What are Proofs?}
% Deductive arguments
In life, we are given bits of information about the world around us.
Statements like ``it is raining in Toronto today", ``if you are shorter than 147
centimeters, then you cannot enter", or ``every person has two biological
parents" is a statement that is either true, or false, exclusively. Presented
with starting information, we are concerned with \emph{reasoning} through
whether another statement is true or false, consistent with our start.

\begin{expl}
  Suppose the second statement in the previous paragraph is true, and
  suppose that Ellie is 170 centimeters tall, Jack is taller than Ellie, and Sam
  is shorter than Jack.

  We are concerned with whether it is true that (a) Ellie can enter (b) Jack
  can enter and (c) Sam can enter.

  \begin{itemize}
    \item Since Ellie is taller than 147 centimeters, she can enter.
    \item Since Jack is taller than Ellie, Jack's height is greater than
          Ellie's height. Therefore, Jack can enter.
    \item We cannot make any conclusion on whether Sam can enter.
  \end{itemize}

  Now, suppose that Ellie is exactly 147 centimeters tall. Then both Ellie and
  Jack can enter, but not Sam. If Ellie is any shorter, then Ellie and Sam may
  not enter, and we have no conclusion on whether Jack can enter.
\end{expl}
To prove a statement is to \emph{formalize} our reasoning about it. We assume a
set of statements which we deem are true (assumptions), and we form a series of
steps that logically follow from the previous step (deductions). In the next
example, we will prefix our assumptions by \assump{} and our deductions by
\deduct.

% Example: Deductive Arguments
\begin{expl}
\assump{} In the land of Knights and Knaves, knights always tell the truth and
knaves always tell lies. People are either knights, or they are knaves in this
land.

You meet two people, Alice and Bob (henceforth referred to as $A$ and $B$,
respectively.) \assump{} $A$ says ``Both of us are knaves."

We wish to determine who is a knight and who is a knave.

\begin{itemize}
  \item[\deduct] Suppose that $A$ is a knight. If $A$ is a knight, then the
        statement ``Both of us are knaves" is a lie (since $A$ is a knight and
        therefore cannot be a knave).
  \item[\deduct] Since knights cannot tell a lie, and $A$ told a lie, then $A$
        must not have been a knight in the first place.
  \item[\deduct] Therefore, $A$ cannot be a knight.
  \item[\deduct] Since $A$ cannot be a knight, then $A$ is a knave.
  \item[\deduct] Since knaves always tell lies, the statement ``Both of us are
        knaves" is a false statement (a lie).
  \item[\deduct] The only way for this statement to be false is when $B$ is a
       knight.
  \item[\deduct] Therefore, $B$ must be a knight. \newline \hrulesmall
  \item[\deduct] Therefore, $A$ is a knave and $B$ is a knight.
\end{itemize}
\end{expl}

Given a statement $S$, a proof for $S$ is a convincing structured argument that
shows $S$ is true. The structure of a proof uses knowledge that we have to
discover facts about that knowledge that we did not previously know. Naturally,
we can use these facts to prove more results. The accumulation of these facts
allows us to develop a body of statements as a mathematical system.

\todo{System Tree Diagram}

This document will discuss a handful of different ways to prove some statement
$S$. Like tools in a toolbox, there are a variety of proof techniques wherein
different scenarios are more effective in proving $S$ than others; indeed, there
is a craftsmanship and beauty in the way for which someone proves $S$.

% allows us to show some perhaps counter-intuitive relationships
The product of a (successful) proof allows us to definitively state whether some
statement is true or false. This allows us to assert some perhaps
counter-intuitive results:
\begin{claim}
  The infinite decimal number $0.999...$ (which we abbreviate to
  $0.\bar{9}$) is equal to $1$.
  \begin{proof}
    We examine properties of and manipulate the number $0.\bar{9}$ to achieve
    the desired equality. Let $n = 0.\bar{9}$. We deduce:
    \begin{align}
       n  &= 0.\bar{9} & \justify{Definition}                             \\
       10n &= 9.\bar{9}   											                          \\
       10n - (n) &= 9.\bar{9} - (0.\bar{9}) & \justify{Definition of $n$} \label{intro: eq: nsubstract} \\
       9n &= 9                                                            \label{intro: eq: 9nwhole}\\
        n &= 1
    \end{align}
    And thus, we see from the equivalence of the first and last equations that
    $0.999 = 1$, as claimed.
  \end{proof}
\end{claim}

Note that the term ``convincing"  stated is extremely important here.
We provide an alternative proof to the claim:
\begin{proof}
  \begin{align}
    3 \cdot 1/3 &= 1/3 + 1/3 + 1/3 \\
                &= 0.\bar{3} + 0.\bar{3} + 0.\bar{3} & \justify{Decimal representation of 1/3} \\
                &= 0.\bar{9}
  \end{align}
  But $3 \cdot 1/3 = 3/3 = 1$.
\end{proof}

Proofs are segregated into two groups: \emph{Direct} and \emph{Indirect}.
We review some fundamental concepts of mathematics before diving in these
techniques.
\end{document}
