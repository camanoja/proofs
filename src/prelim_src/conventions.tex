\section{Conventions}
  % assigning a number to a letter
  \subsubsection{Assigning a Number to a Letter}
  The mathematical statement:
  $$x = 2.2$$
  \textbf{assigns} the number $2.2$ to the letter $x$, and allows us to use $x$
  in place of $2.2$. If we expect the value of $x$ to take different values,
  then we call $x$ a variable. If we expect the value of $x$ to stay the same,
  then we call $x$ a constant.

  This notation is especially useful if we want to store the effect of an
  operation without explicitly calculating it. This allows us a simple yet
  powerful freedom to algibraically manipulate our values:

  \begin{expl}{\todo{Variable Manipulation}}
    Let $x = {5}/{2}$. Then $x^2$.
  \end{expl}

  \subsubsection{Assigning a Mathematical Object to a Letter}
  In the same way as numbers, we can also use variables to refer to
  mathematical objects other than numbers. We may say that $x = (2, 9)$,
  $y = 5 + 3i$, $z = f$ (where $f(x) = x^2$), et cetera.
