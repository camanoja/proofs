\section{Properties of Equality}
Suppose that two mathematical objects (numbers, ordered pairs, matrices,
...) $a$ and $b$ are the same object, or equal. We express this as $a = b$.

Then to express what we mean by $a$ is the same as $b$, we assume that the
following statements are true:
\begin{enumerate}[label=(\alph*)]
  \item If $a = b$ and $b = c$, then $a = c$. This property is called
  \textbf{transitivity}.
  \item If $a = b$ then $b = a$. This property is called \textbf{symmetry}.
  \item $a = a$ ($a$ is equal to itself). This property is called
  \textbf{reflexivity}.
\end{enumerate}

% TODO: equality is important because in algebra we often add and subtract
% from both sides of an equation
\section{Inequalities}
In addition to equality between two numbers $a$ and $b$, we can express
relationships between them in the form of inequality operators:

\begin{itemize}
  \item We say that $a < b$ if $b = a + c$, for some \emph{positive}
  number $c$.
  \item We say that $a > b$ if $b < a$.
  \item If either of the above are satisfied, then $a \neq b$.
\end{itemize}


% a - a = 0
\begin{theorem}
  Let $a, b \in \reals$ such that $a = b$. Then $a - b = 0$
\end{theorem}
\begin{proof}
  If $a - b \neq 0$, then either $a - b > 0$ or $a - b < 0$.

  So, if $a - b > 0$, then $a = b + c$, for some $c \in \reals, c > 0$. That is
  to say $a > b$, which contradicts our assumption that $a = b$. In the other
  case, $b = a + c$, meaning that $b > a$, giving us another contradiction.
\end{proof}
