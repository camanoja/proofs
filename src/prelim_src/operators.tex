\newcommand{\op}{\circ}
\newcommand{\ftoperandfn}{Operators can be thought of as shorthand for
functions. For example, the $-$ operator is shorthand to the function
$n(x) = -x$, and $+$ is a shorthand for $a(x, y) = x + y$.}

\newcommand{\ftoperanddist}{It is not always the case that
$a \op_1 (b \op_2 c) = (b \op_2 c) \op_1 a$. If we need to make a distinction
between the two forms, we will refer to the left form as left distributivity,
and the right form as right distributivity.}

\section{Operator Properties}
Expressions like $-7$, $2.2 + 3$, $2.2 - 3$ can be deconstructed into two
parts: \emph{operators} and \emph{operands}. In this context, $+$ is an
operator that acts on the operands $2.2$ and $3$ in the set of real numbers.
Since $+$ acts on two numbers, we call it a binary operator. The symbol $-$ is
an operator that acts on the single integer $7$, so we see it as a unary
operator.

More generally, let $\op$ denote some operation on a set of one or more
mathematical objects $S$. Then we say that $\op$ is defined over $S$.

Let $x, y \in S$.
% unary, binary, ternary operations
If $\op x$ is a valid operation, then we call $\op$ a unary operator. If
$x \op y$ is a valid operation, then we call $\op$ a binary operator.
Unless conventionally specified by BEDMAS / PEMDAS, we will use parentheses ( )
and square brackets $[$ {} $]$ to denote operator precedence.

We describe a few useful properties of binary operators:
\begin{itemize}
  % transitivity
  \item If for every $a, b \in S$ we have $a \op b = b \op a$, then we say
  that $\op$ is \emph{transitive}.
  % associativity
  \item If for every $a, b, c \in S$ we have $(a \op b) \op c =
  a \op (b \op c)$, then we say that $\op$ is \emph{associative}.
  % distributivity
  \item Let $\op_1$ and $\op_2$ be defined on the set $A$.
  respectively. If $a \op_1 (b \op_2 c) = (a \op_1 b) \op_2 (a \op_1 c) =
  (b \op_2 c) \op_1 a$ then we say that $\op_1$ \emph{distributes over}
  $\op_2$.
\end{itemize}

We will assume the following of our binary operators:
\begin{prop}
  Common Properties of Binary Operators
\end{prop}
\begin{itemize}
  \item $+$ (addition) and $\cdot$ (multiplication) is transitive and
   associative over $\naturals, \ints, \rationals, \irrationals$ and $\reals$.
  \item $-$ is associative over $\naturals, \ints, \rationals, \irrationals$
   and $\reals$.
\end{itemize}

We will define the division operation in the following way. Given 
$p, q \in \reals$ where $q \neq 0$ we say that $p/q = r$ is equivalent
to $p = rq$ for some $r \in \reals$. 


\section{Operative Closure}
We say that a set $S$ is \textbf{closed} under a certain operation if for
\emph{any} two elements $a, b \in S$, the resulting object $c$ that is the
result of the operation between $a$ and $b$ \emph{is also} in $S$. The
following properties are true:

\begin{itemize}
  \item $\naturals$ is closed under addition and multiplication, but not
  subtraction and division.
  \item $\ints$ is closed under addition, multiplication and subtraction, but
  not division.
\end{itemize}

$\naturals$ is not closed under subtraction, as
$(8) - (9) = -1 \not \in \naturals$ and $\ints$ is not closed under division
as $(3) \div (2) = 1.5 \not\in \ints$.
