\section{Sets}
We define a set to be an unordered (and possibly infinite) collection of
unique objects called elements, or members. We denote a set by surrounding
its elements by curly braces, as in the following example:

\begin{expl}{Finite Sets}
  \begin{itemize}
    \item Let $S_1$ be the set of non-negative integers which are less than
    10. Then $S_1 = \toset{0, 1, 2, ..., 9}$, and $S_1$ has 10 elements.

    \item Let $S_2$ be the set of integers whose square is equal to $16$.
    Then $S_2 = \toset{-4, 4}$, and $S_2$ has 2 elements.

    \item Let $S_3$ be the set of odd numbers that are evenly divisible by 2.
    Then $S_3 = \toset{}$, as $S_3$ has 0 elements. A set with no elements is
    called the \textbf{empty set}, and is denoted with $\emptyset$.
  \end{itemize}
\end{expl}

A set is unordered in the sense that the order for which we list its elements
does not affect it. So, from the above example, $S_2 = \toset{4, -4}$ as
well. A set is a collection of unique objects in the sense that the number of
times that an element appears in the set does not affect it. So, $S_2 =
\toset{-4, -4, 4}$. If $a$ is a member of a set $S$, we denote this
by $a \in S$. If $a$ is not a member of $S$, then we denote this as
$a \not \in S$.


We can describe a set in two ways. One way, called an extensive description,
is to write down all of its elements as we have done in the above
example. The other way, called an intensive description, is to write down the
form of an arbitrary element in the set, followed by a rule that the form
takes. Symbolically:
$$ \toset{ \lrangle{\text{form}} : \lrangle{\text{rule}} } $$

The latter description is helpful when describing infinite sets.

\begin{expl}{Infinite Sets}
  \begin{itemize}
    \item Let $S_4$ be the set of integers. Then
    $S_4 = \toset{..., -2, -1, 0, 1, 2, ...}$. We denote the integer set
    as $\ints$.

    \item Let $S_5$ be the set of all perfect squares. Then
    $S_5 = \toset{0, 1, 4, ...}$ (extensive description) and equivalently,
    $S_5 = \toset{n^2 : n \in \ints }$ (intensive description)

    \item Let $S_6$ be the set of values $x$ that satisfy $-2 \leq x$ and
    $x \leq -1$. Then $S_6 = \toset{x : x \in \reals, -2 \leq x \leq 1}$,
    where the comma in the rule is a shorthand for ``and".

    We call $S_6$ an interval, and we denote it by $[-2, 1]$. If instead we
    required $-2 < x \leq 1$ then we would denote it by $(-2, 1]$.

    \item Let $S_7$ be the set of all points $(x, y)$ on the cartesian plane
    that lie on the unit circle. Then $S_7 = \toset{ (x, y) :
    x^2 + y^2 = 1 \text{ and } x, y \in [-1, 1]}$.
  \end{itemize}
\end{expl}

Some important number sets are defined in section \ref{subsec: defnumbersets}.

\section{Operations on Sets}
% union, concatenation, subtraction
