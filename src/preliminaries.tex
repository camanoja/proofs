\documentclass[../proofs.tex]{subfiles}

\begin{document}
\chapter{Preliminaries}
  \section{Definitions}
In order to formalize our reasoning on objects which have properties like
numbers, vectors, or shapes, we need to first make \textbf{mathematically
clear} what we mean when we talk about numbers, vectors or shapes.

\begin{expl}{Definition of the even and odd numbers}
  Given the non-negative integers $0, 1, 2, \ldots$, we want to characterize
  the even numbers from the odd numbers.

  The even numbers are those numbers that are evenly divisible by 2:
  $$(0, 2, 4, 6 \ldots) = (2 \cdot 0, 2 \cdot 1, 2 \cdot 2, 2 \cdot 3 \ldots)$$.
  We define even numbers as those numbers of the form $2 \cdot k$, where $k$
  is some non-negative whole number.


  The odd numbers are those numbers not evenly divisible by 2:
  \begin{align*}
    (1, 3, 5, \ldots) &= (0 + 1, 2 + 1, 4 + 1, \ldots) \\
                      &= (2 \cdot 0 + 1, 2 \cdot 1 + 1, 2 \cdot 2 + 1, \ldots)
  \end{align*}
  We define the odd numbers are defined by those numbers of the form
  $2 \cdot k + 1$, where $k$ is some non-negative whole number.
\end{expl}

A definition of something formalizes precisely what characterises that
object. If we are to show that some object $O$ fits our definition, then
we must show that $O$ has exactly those properties listed in the definition.

\begin{expl}{Definition of even and odd functions}
  As stated in the introduction, a function is a ``mathematical
  machine" that recieves an input and returns an output that is unique for
  that input. That is, a function $f$ satisfies the property:
  $$\text{If } a = b, \text{ then } f(a) = f(b).$$
  The complete set of input values that $f$ can take is called its domain and
  we denote it by $\dom(f)$, and the complete set of $f$'s output values is
  called its range, and we denote it by $\ran(f)$.


  We will say that a function $f_E$ is an even function if it is a function
  and it satisfies the following property for all input values in $\dom(f_E)$:
  $$f_E(-x) = f_E(x)$$

  We will say that a function $f_O$ is an odd function if it is a function and
  it satisfies the following for all input values in $\dom(f_O)$:
  $$f_O(-x) = -f_O(x)$$

  Examples of even functions are $f(x) = x^2, g(x) = \ln(|x|)$ and
  $h(x) = 1/\sqrt{1-x^2}$. Examples of odd functions are $f'(x) = \tan(x)$,
  $g'(x) = x^3$ and $h'(x) = 0$, as is easily verifiable.
\end{expl}


% definitions, formalization, deductions, conclusions

  \section{Conventions}
  % assigning a number to a letter
  \subsubsection{Assigning a Number to a Letter}
  The mathematical statement:
  $$x = 2.2$$
  \textbf{assigns} the number $2.2$ to the letter $x$, and allows us to use $x$
  in place of $2.2$. If we expect the value of $x$ to take on different values,
  then we call $x$ a variable. If we require the value of $x$ to stay the same,
  then we call $x$ a constant.

  This notation is especially useful if we want to store the effect of an
  operation without explicitly calculating it. This allows us a simple yet
  powerful freedom to algebraically manipulate our values:

  \begin{expl}{Variable Manipulation}
    Let $a = p/q$ and $b = r/s$ where $q, r, s$ are not equal to $0$.
    We wish to find the value of $a \div b$. Assign $D = a/b$. We have:
    \begin{align}
      D       &= a/b \\
      bD      &= p/q     & \justify{multiplying both sides by $b$}\\
      (r/s)D  &= p/q \\
      rD      &= p/q \cdot s, &\justify{multiplying both sides by $s$} \\
      D       &= (p/q) \cdot (s/r) & \justify{multiplying by $1/r$}
    \end{align}

    So $\displaystyle{\frac{p}{q} \div \frac{r}{s} =
     \frac{p}{q} \cdot \frac{s}{r}}$,
    as expected.
    \begin{comment}
    We wish to find the value of the infinitely nested square root
    $\sqrt{6\sqrt{6 \sqrt{6 ...}}}$. Let $x = \sqrt{6\sqrt{6 \sqrt{6 ...}}}$.
    Then:
    \begin{align*}
      x^2 &= 6 \sqrt{6 \sqrt{6 ...}} \\
      x^2 &= 6x \\
      x^2 - 6x &= 0 \\
      x(x-6) &= 0
    \end{align*}

    So either $x=0$ or $x=6$. Since
    $\sqrt{6 \sqrt{6 ...}} > \sqrt{6 \sqrt{0}} = 0$ we must have that $x = 6$.
    \end{comment}
  \end{expl}

  \subsubsection{Assigning a Mathematical Object to a Letter}
  In the same way as numbers, we can also use variables to refer to
  mathematical objects other than numbers. We may say that $x = (2, 9)$,
  $y = 5 + 3i$, $z = f$ (where $f(x) = x^2$), et cetera.

  \subsubsection{Quantifiers}
  % like a game
  % for all - meaning someone can choose any value
  % there exists - meaning we choose that value
  % such that : "with the property"

  % on language; common phrases
  % holds/does not hold
  % ``we have"
  % it follows that

  \section{Properties of Equality}
Suppose that two mathematical objects (numbers, ordered pairs, matrices,
...) $a$ and $b$ are the same object, or equal. We express this as $a = b$.

Then to express what we mean by $a$ is the same as $b$, we assume that the
following statements are true:
\begin{enumerate}[label=(\alph*)]
  \item If $a = b$ and $b = c$, then $a = c$. This property is called
  \textbf{transitivity}.
  \item If $a = b$ then $b = a$. This property is called \textbf{symmetry}.
  \item $a = a$ ($a$ is equal to itself). This property is called
  \textbf{reflexivity}.
\end{enumerate}

% TODO: equality is important because in algebra we often add and subtract
% from both sides of an equation
\section{Inequalities}
In addition to equality between two numbers $a$ and $b$, we can express
relationships between them in the form of inequality operators:

\begin{itemize}
  \item We say that $a < b$ if $b = a + c$, for some \emph{positive}
  number $c$.
  \item We say that $a > b$ if $b < a$.
  \item If either of the above are satisfied, then $a \neq b$.
\end{itemize}


% a - a = 0
\begin{theorem}
  Let $a, b \in \reals$ such that $a = b$. Then $a - b = 0$
\end{theorem}
\begin{proof}
  If $a - b \neq 0$, then either $a - b > 0$ or $a - b < 0$.

  So, if $a - b > 0$, then $a = b + c$, for some $c \in \reals, c > 0$. That is
  to say $a > b$, which contradicts our assumption that $a = b$. In the other
  case, $b = a + c$, meaning that $b > a$, giving us another contradiction.
\end{proof}

  \section{Sets}
We define a set to be an unordered (and possibly infinite) collection of
unique objects called elements, or members. We denote a set by surrounding
its elements by curly braces, as in the following example:

\begin{expl}{Finite Sets}
  \begin{itemize}
    \item Let $S_1$ be the set of non-negative integers which are less than
    10. Then $S_1 = \toset{0, 1, 2, ..., 9}$, and $S_1$ has 10 elements.

    \item Let $S_2$ be the set of integers whose square is equal to $16$.
    Then $S_2 = \toset{-4, 4}$, and $S_2$ has 2 elements.

    \item Let $S_3$ be the set of odd numbers that are evenly divisible by 2.
    Then $S_3 = \toset{}$, as $S_3$ has 0 elements. A set with no elements is
    called the \textbf{empty set}, and is denoted with $\emptyset$.
  \end{itemize}
\end{expl}

A set is unordered in the sense that the order for which we list its elements
does not affect it. So, from the above example, $S_2 = \toset{4, -4}$ as
well. A set is a collection of unique objects in the sense that the number of
times that an element appears in the set does not affect it. So, $S_2 =
\toset{-4, -4, 4}$. If $a$ is a member of a set $S$, we denote this
by $a \in S$. If $a$ is not a member of $S$, then we denote this as
$a \not \in S$.


We can describe a set in two ways. One way, called an extensive description,
is to write down all of its elements as we have done in the above
example. The other way, called an intensive description, is to write down the
form of an arbitrary element in the set, followed by a rule that the form
takes. Symbolically:
$$ \toset{ \lrangle{\text{form}} : \lrangle{\text{rule}} } $$

The latter description is helpful when describing infinite sets.

\begin{expl}{Infinite Sets}
  \begin{itemize}
    \item Let $S_4$ be the set of integers. Then
    $S_4 = \toset{..., -2, -1, 0, 1, 2, ...}$. We denote the integer set
    as $\ints$.

    \item Let $S_5$ be the set of all perfect squares. Then
    $S_5 = \toset{0, 1, 4, ...}$ (extensive description) and equivalently,
    $S_5 = \toset{n^2 : n \in \ints }$ (intensive description)

    \item Let $S_6$ be the set of values $x$ that satisfy $-2 \leq x$ and
    $x \leq -1$. Then $S_6 = \toset{x : x \in \reals, -2 \leq x \leq 1}$,
    where the comma in the rule is a shorthand for ``and".

    We call $S_6$ an interval, and we denote it by $[-2, 1]$. If instead we
    required $-2 < x \leq 1$ then we would denote it by $(-2, 1]$.

    \item Let $S_7$ be the set of all points $(x, y)$ on the cartesian plane
    that lie on the unit circle. Then $S_7 = \toset{ (x, y) :
    x^2 + y^2 = 1 \text{ and } x, y \in [-1, 1]}$.
  \end{itemize}
\end{expl}

Some important number sets are defined in section \ref{subsec: defnumbersets}.

\section{Operations on Sets}
% union, concatenation, subtraction

  \newcommand{\negfrac}[1]{\frac{ (-1)^{#1} }{ #1 } }

\section{The Sum Operator}
In this booklet, we will encounter additions of multiple summands such as:
$$ S = 1 + 4 + 9 + ... + 100. $$
Explicitly writing each summand is laborious, so we express this sum
with the shorthand: $$ S = \sum_{i=1}^{10}{i^2}. $$

The symbol $\sum$ (big sigma) is known as the \textbf{sum operator}. In
general, the sum operator takes the form:

$$ \sum_{i=a}^{b} f(i) = f(a) + f(a+1) + \ldots + f(b-1) + f(b),$$

For $a, b \in \naturals$ and $a \leq b$. We call $i$ the index, which ranges
sequentially from $a$ to $b$. We call the function $f$ the sum term, where
the value $f(i)$ exists for all values that $i$ takes.

\begin{expl}{Finite Sums}
  \begin{enumerate}
    \item Let $a = 1; b = 4; f(i) = (-1)^i/i$. We have:
    \begin{align}
      \sum_{i=0}^{4} \frac{ (-1)^i }{ i }
        &= \negfrac{1} + \negfrac{2}
        + \negfrac{3} + \negfrac{4}  \label{prelim: eq: sumcos} \\
        &= - 1 + \frac{1}{2} - \frac{1}{3} + \frac{1}{4} \\
        &= -\frac{1}{2} - \frac{1}{3} + \frac{1}{4} \\
        &= -\frac{7}{12}
    \end{align}

    \item Let $a = -2, b = 2; f(x) = 1/x$. We have:
    \begin{align}
      \sum_{x=-2}^{2} \frac{1}{x}
        &= \frac{1}{-2} + \frac{1}{-1} + \frac{1}{0} + \frac{1}{1}
        + \frac{1}{2} \\
        &= \frac{1}{0}, & \text{ simplifying }
    \end{align}
    So this sum is undefined, as $f(x)$ is undefined when $x = 0$.
  \end{enumerate}
\end{expl}

It should be noted that there is usually no unique way to describe a sum in
sum notation. The first sum could have been expressed as
$\sum_{i=-2}^{6}{(i+2)^2}$, and the sum in expression
(\ref{prelim: eq: sumcos}) could have been expressed as
$\sum_{i=0}^{4}{ \cos(i\pi) / {i!} }.$

% Properties of the sum operator.
All the familliar properties of regular sums apply to $\sum$.
\begin{prop} (Scalar Distributivity)
  Let $c \in \reals$. Then $$c \sum_{i=a}^{b} f(i) =
  \sum_{i=a}^{b} c \cdot f(i).$$
\end{prop}

\begin{prop} (Combination)
  Let $a, b \in \ints$ such that $a \leq b$. Then:
  $$\sum_{i=a}^{b} (f(i) + g(i))
  = \sum_{i=a}^{b} f(i) + \sum_{i=a}^{b} g(i).$$
\end{prop}

  \newcommand{\op}{\circ}

\section{Operator Properties}
Let $\op$ denote some operation on a set of one or more mathematical
objects $S$. Let $x, y, z \in S$.

% unary, binary, ternary operations
If $\op x$ is a valid operation, then we call $\op$ a unary operator. If
$x \op y$ is a valid operation, then we call $\op$ a binary operator.

% associativity

% reflexivity

% transitivity


\section{Operative Closure}
We say that a set $S$ is \textbf{closed} under a certain operation if for
\emph{any} two elements $a, b \in S$, the resulting object $c$ that is the
result of the operation between $a$ and $b$ \emph{is also} in $S$. The
following properties are true:

\begin{itemize}
  \item $\naturals$ is closed under addition and multiplication, but not
  subtraction and division.
  \item $\ints$ is closed under addition, multiplication and subtraction, but
  not division.
\end{itemize}

$\naturals$ is not closed under subtraction, as
$(8) - (9) = -1 \not \in \naturals$ and $\ints$ is not closed under division
as $(3) \div (2) = 1.5 \not\in \ints$.


  \section{Common Definitions}
  \subsection{Number Sets}{\label{subsec: defnumbersets}}
  \begin{itemize}
    \item The set $\naturals = \toset{0, 1, 2, ...}$ is called the natural
    number set.
    \item The set $\ints = \toset{ ..., -2, -1, 0, 1, 2, ... }$ is called
    the integer set.
    \item The set of all numbers that are equal to a ratio of any two
    integers $\rationals = \toset{p/q : p, q \in \ints, q \neq 0}$ is called
    the rational number set.
    \item The set of all numbers that are not equal to any ratio of two
    integers $\irrationals = \toset{..., \phi, \sqrt{2}, e, \pi, ...}$ is
    called the irrational number set.
    \item The combined set of the rationals and irrationals
    $\reals = \rationals \cup \irrationals$ is called the real number set.
  \end{itemize}

  \section{Optional: A Primer on Symbolic Logic}


\end{document}
