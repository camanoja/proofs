\documentclass[../proofs.tex]{subfiles}

\begin{document}
\chapter{Preliminaries}
We review and define some necessary concepts before we begin our discussion of
proofs.

\section{Definitions}
In order to formalize our reasoning on objects which have properties like
numbers, vectors, shapes, etc., we need to first make \emph{mathematically
clear} what we mean when we talk about numbers, vectors or shapes.

\begin{expl}{Definition of the even and odd numbers}
  Given the non-negative integers $0, 1, 2, \ldots$, we want to characterize
  the even numbers from the odd numbers.

  The even numbers are those numbers that are evenly divisible by 2:
  $$0, 2, 4, 6 \ldots = 2 \cdot 0, 2 \cdot 1, 2 \cdot 2, 2 \cdot 3 \ldots$$.
  We define even numbers as those numbers of the form $2 \cdot k$, where $k$
  is some non-negative whole number.


  The odd numbers are those numbers not evenly divisible by 2:
  \begin{align*}
    1, 3, 5, \ldots &= 0 + 1, 2 + 1, 4 + 1 \\
                    &= 2 \cdot 0 + 1, 2 \cdot 1 + 1, 2 \cdot 2 + 1
  \end{align*}
  We define the odd numbers are defined by those numbers of the form
  $2 \cdot k + 1$, where $k$ is some non-negative whole number.
\end{expl}

 Definitions for objects serve as our foundation as they characterize what we
 mean by those objects.

 % definitions, formalization, deductions, conclusions

\section{Conventions}
  % assigning a number to a letter
  \subsubsection{Assigning a Number to a Letter}
  The mathematical statement $n = 2.2$ \emph{assigns} the number $2.2$ to the
  letter $n$, and allows us to use the $n$ in replacement of where the number
  would exist. The $n$ is known as a variable. This assignment usually exists
  for the sake of brevity and generalization.

  The context of the usage of the variable defines the operation to take place.
  For example, $n = m$ may either state that we assign the value of $m$ to $n$,
  or we are asserting that $n$ is \emph{equal} to $m$.

  \subsubsection{Assigning a Mathematical Object to a Letter}
  In the same way as numbers, we can also use variables to describe mathematical
  objects other than numbers. We may say that $x = (2, 9)$, $y = 5 + 3i$,
  $z = f$ (where $f$ is a function), etc.

\begin{comment}
  % theorems, lemmas and corollaries
  \subsubsection{Theorems vs. Lemmas vs. Corollaries}
  What defines a theorem is obscure, but the general rule is as follows.
  A theorem describes some sort of large or powerful result, while a lemma
  describes a subordinate result (sometimes called a ``stepping stone") used to
  achieve a theorem. A corollary is a result that is an immediate consequence
  of a theorem.
\end{comment}

\section{Properties of Equality}
  Suppose that two mathematical objects (numbers, ordered pairs, matrices, etc.)
  $a$ and $b$ are equal. We express this as $a = b$.

  Then, we assume that the following statements are true:
  \begin{enumerate}[label=(\alph*)]
    \item The equality is \emph{transitive}. That is, if $a = b$ and $b = c$,
    then $a = c$. For example, we know that $2 = 4/2$. We know also that
    $2 = \sqrt[3]{8}$. Hence, $\frac{4}{2} = \sqrt[3]{8}$.

    \item The equality is \emph{symmetric}. That is, if $a = b$ then $b = a$.
    For example, $0.49 = 49/100$ and $49/100 = 0.49$

    \item The equality is \emph{reflexive}. That is, $a = a$ (any object is
    equal to itself). For example, $5 \cdot 10 = 5 \cdot 10.$
  \end{enumerate}

\section{Number Sets}
  \subsubsection{Definitions}
  A set is an unordered (and possibly infinite) collection of numbers.
  \begin{enumerate}
    \item $\naturals$: The set containing the non-negative whole numbers. In
    set notation, we say $\naturals = \toset{0, 1, 2, \ldots}$.

    \item $\ints$: The set containing the integers. $\ints = \toset{\ldots, -2,
    -1, 0, 1, -2, \ldots}$.

    \item $\rationals$: The set containing all numbers which can be expressed as
    a fraction of two integers:
    $\rationals = \toset{\frac{p}{q}: p, q \in \ints; q \neq 0}$.

    \item $\bar{\rationals}$: The set containing the numbers for which cannot
    be expressed as a fraction of two integers:
    $\bar{\rationals} = \toset{\ldots, \pi, e, \sqrt{2}, \phi, \ldots}$.

    \item $\reals$: The combined set of the $\rationals$ and $\bar{\rationals}$:
    $\reals = \rationals \cup \bar{\rationals}$
  \end{enumerate}

  $\naturals$ is known as the natural numbers, $\ints$ is known as the integers,
  $\rationals$ is known as the rational numbers, $\bar{\rationals}$ is known as
  the irrational numbers, and $\reals$ is known as the real numbers.

  If a number $n$ exists within a set $S$, then we say that $n$ is a
  \emph{member} of $S$. In mathematical notation, $n \in S$. As an example, we
  have that $-5 \in \ints$ but $-5 \not\in \naturals$.

  \subsubsection{Closure}
  We say that a set $S$ is \emph{closed} under a certain operation if for
  \emph{any} two elements $a, b \in S$, the resulting object $c$ that is the
  result of the operation between $a$ and $b$ \emph{is also} in $S$. The
  following properties are true:

  \begin{itemize}
    \item $\naturals$ is closed under addition and multiplication, but not
    subtraction and division.
    \item $\ints$ is closed under addition, multiplication and subtraction, but
    not division.
  \end{itemize}

  We see that $\naturals$ is not closed under subtraction, as
  $(8) - (9) = -1 \not \in \naturals$ and $\ints$ is not closed under division
  as $(3) \div (2) = 1.5 \not\in \ints $

\section{The Sum Operator}
  \newcommand{\negfrac}[1]{\frac{ (-1)^{#1} }{ #1! } }

  Consider the sum: $$ 4 + 9 + 16 + 25 + ... + 100. $$
  We may express this sum succinctly as: $$ \sum_{i=2}^{10}{i^2}. $$

  This symbol $\sum$ is known as the \emph{sum operator}. In general, the sum
  operator describes the sum:
  $$ \sum_{i=a}^{b}{s_i} = s_a + s_{a+1} + \ldots + s_{b-1} + s_b. $$

  The variable $i$ is known as the \emph{index}, taking the values
  $a, a+1, \ldots, b-1, b$ where $a \leq b$. $s_i$ is known as the
  \emph{sum term}, a term for whose value depends on the value $i$.

  For example, for $a = 0; b = 4; s_i = (-1)^i/i!$ we have:
  \begin{align}
    \sum_{i=0}^{4}{\frac{(-1)^i}{i!}}
      &= \negfrac{0} + \negfrac{1} + \negfrac{2}
      + \negfrac{3} + \negfrac{4}  \label{prelim: eq: sumcos} \\
      &= 1 - 1 + \frac{1}{2} - \frac{1}{6} + \frac{1}{24} \\
      &= \frac{1}{2} - \frac{1}{6} + \frac{1}{24} \\
      &= \frac{3}{8}
  \end{align}

  It should be noted that there is usually no unique way to describe a sum in
  sum notation. The first sum could have been expressed as
  $\sum_{i=0}^{8}{(i+2)^2}$, and the sum in \ref{prelim: eq: sumcos} could have
  been expressed as $\sum_{i=0}^{4}{ \frac{\cos{i\pi}} {i!} }.$

  \section{Operator Properties}
  \newcommand{\op}{\circ}
  Let $\op$ denote some operation on a set of one or more mathematical
  objects $S$. Let $x, y, z \in S$.

  % unary, binary, ternary operations
  If $\op x$ is a valid operation, then we call $\op$ a unary operator. If
  $x \op y$ is a valid operation, then we call $\op$ a binary operator.

  % associativity

  % reflexivity

  % transitivity

  \section{Common Definitions}


\end{document}
