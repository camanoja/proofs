\documentclass[../proofs.tex]{subfiles}

\begin{document}
\chapter{Questions}
	\begin{enumerate}
		\item Let $n \in \mathbb{N}, n > 2$ and $n | 2 = 0$ (that is, there is no remainder when dividing $n$ by 2). Prove that $n$ is not prime.
		\item Prove directly that the sum of an even integer and another even integer (not necessarily the same integer) is even.
			\subitem --- What can you conclude about the sum of an odd integer and an even integer? Can you directly prove it?
		\item \challenge Prove that any odd number is equal to a difference of squares\footnote{A difference of squares is of the form $(a^2 - b^2)$, where $a, b \in \mathbb{Z}$}.
			\subitem --- Hint: This difference of squares is not unique.
			\subitem --- Hint: This proof is facilitated by \emph{constructing} the difference of squares.
		\item (\emph{Binomial Expansion Theorem of degree} 2) Prove geometrically that $(a+b)^2 = a^2 + 2ab + b^2$.
		\item Let $a, b, c, d \in \mathbb{R}$; where $b, c, d \neq 0$. Prove that $\frac{a}{b} \div \frac{c}{d} = \frac{a}{b} \cdot \frac{d}{c}$. (You may assume that $\frac{a}{b} \cdot \frac{b}{a} = 1$, under the appropriate conditions.)
		\item Prove $\sum_{i = 1}^{n} i^2 = \frac{n(n+1)(2n+1)}{6}$, by induction.
		\item Prove $\sum_{i = 1}^{n} i^3 = (\sum_{i = 1}^{n} i)^2 $.
		\item \challenge Prove, by induction that the degree-sum of the internal angles of $n$-sided polygon is equal to $(n-2)*180$:
			\begin{itemize}
				\item (\emph{Base Case}): What is the simplest shape for which this fact can be verified?
				% index = # sides, call it k?
				\item (\emph{Induction Hypothesis}): What may we use for our index value?
				\item (\emph{Induction Step}): Construct an arbitrary shape $S$, that has $k$ sides.
				\subitem{--- Argue there are 2 ways of creating a new side from an arbitrary side $s_i$ in $S$. Name this new shape $S\prime$.}
				\subitem{--- For each of the cases detailed above, verify that $S\prime$ satisfies the relevant angle equation, keeping in mind the number of 											sides in $S\prime$.}
			\end{itemize}

			\item \challenge Using geometric arguments, directly prove the \emph{Pythagorean Theorem} for a right-angle triangle $\Delta ABC$, where $|AB|=|BC|$ (that is, where the base and height of $\Delta ABC$ are equal.)
		\item \challenge
			\begin{enumerate}
				\item Consider the \emph{last digit} of $2^2, 2^6, \text{ and } 2^{10}$. Do you see a pattern?
				\item Is this pattern true \emph{for all} $2^n, n\in\mathbb{N}$? If not, then what should $n$ be? \label{item: n_form}
				\item Prove this pattern for all numbers $2^n$, for whatever $n$ must be in \ref{item: n_form}).
			\end{enumerate}

		\end{enumerate}
\end{document}